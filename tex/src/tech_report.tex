\begin{center}
\bfseries{\large ТЕХНИЧЕСКИЙ ОТЧЁТ ПО ПРАКТИКЕ}
\end{center}

\section*{Архитектура}
\textbf{Парсер:} операционная система Mac OS, framework Scrapy (Python версии 2.7), json-формат.
\textbf{Сайт:} операционная система Windows 10, framework Django (Python версии 3.5).
\section*{Описание}
Сперва проводился парсинг условного сайта с помощью Scrapy, выводя данные в файл формата json. Затем файл загружается в папку fixtures, после этого с помощью команды loaddata информация из этого файла заносится в базу данных и при обновлении сайта будет отображена.
\section*{Реализация}
\textbf{Парсер:} сперва создавался проект в котором и лежит весь парсер. Его основой служил так назывемый паук, его код является небольшим поэтому можно указать его основные части [Имя паука, код основной страницы, цикл для обхода одинакового формата данных с помощью указания xpath для каждого Fields() для базы данных, взятого с элемтов html сайта]. \textbf{Сайт:} сначала создавался общий проект Django. Потом внутри него создаётся приложение, которое имело имя portal, в нем, папке urls.py прописывалось ссылки на необходимые страницы для пользователя. В папке view.py создавались необходимые для работы функции. И в папке modals.py были заданы модели классов, определяющие базу данных. В папке templates находились отверстанные шаблоны, отображающиеся в браузере, а в папку fixtures заносились файлы с расширение json для заполнения базы данных.
\section*{Тестирование}
Тестирование парсера проводилось изменением пути xpath до момента, когда не был получен нужный вывод данных. При тестировании сайта происходили попытки добавить новый турнир, новый год, с последующим просмотром выводимых данных.
\section*{Ссылка на GitHub}
https://github.com/DmitriyDergach/Practica2019
\pagebreak
